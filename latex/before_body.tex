% Begin Committee Approval Page -------
\newpage
\thispagestyle{empty}
\begin{center}

\includegraphics[width=0.75in, height=0.75in]{./figures/unr_logos/University Logo RGB_block_n_blue}

THE GRADUATE SCHOOL

\vspace{1em}
We recommend that the thesis \\
prepared under our supervision by\\

\vspace{1em}
\textbf{Alexander D. Knudson}

\vspace{1em}
entitled

\textbf{A Bayesian Multilevel Model for the Psychometric Function using R and Stan}

\vspace{2em}
be accepted in partial fullfilment of the \\
requirements for the degree of

\vspace{1em}
\textbf{Master of Science}

\vspace{1em}
Dr. A. Grant Schissler \\
\textit{Advisor}

\vspace{1em}
Dr. Colin Grudzien\\
\textit{Committee Member}

\vspace{1em}
Dr. Fang Jiang \\
\textit{Graduate School Representative}

\vspace{1em}
David W. Zeh, Ph.D., Dean \\
\textit{Graduate School}

\vspace{1em}
December, 2020
\end{center}
%---------- End Committee Approval Page


% Begin ---------------
\newpage
\setcounter{page}{1} % Begin lower case Roman numerals
\section*{Abstract}

A common neuroscience topic is to determine the temporal order of two stimuli, and is often studied via a logistic model called a psychometric function. The data arises from repeated sampling of subjects across a variety of tasks (stimuli), blocks, and time separations. These studies are often interested in making inferences at the group level (age, gender, etc.) and at an individual level. This hierarchical nesting makes multilevel modeling a natural choice for these data. We describe a principled workflow for model development using domain expertise, regularizing priors, and posterior predictive performance to compare models. We then apply the workflow to produce a novel statistical model for temporal order judgment data by fitting a series of Bayesian models efficiently using Hamiltonian Monte Carlo (HMC) in the R programming language with Stan.
%------------------ End


% Begin ---------------
\setcounter{tocdepth}{1}
\tableofcontents

\listoftables

\listoffigures
%------------------ End
